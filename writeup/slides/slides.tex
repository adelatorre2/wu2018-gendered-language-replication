\documentclass{beamer}
\usetheme{Madrid}
\usepackage{graphicx}
\usepackage{booktabs}
\usepackage{adjustbox}
\usepackage{xcolor}
\usepackage{tikz}
\definecolor{maleBlue}{RGB}{44,123,182}
\definecolor{femalePink}{RGB}{214,96,173}

\title[Wu 2018 Replication]{Replication of Wu (2018): Gendered Language on EJMR}
\author[A. De La Torre]{Alejandro De La Torre}
\date{\today}

\begin{document}

\begin{frame}
  \titlepage
\end{frame}

\begin{frame}{Paper overview}
\begin{itemize}
  \item Question: How are women and men discussed on EJMR, and do the words differ systematically?
  \item Approach: Lasso logistic models on word counts to identify gendered language.
  \item Replication targets: Table 1, Table 2, Figure 1 from Wu (2018).
\end{itemize}
\end{frame}

\begin{frame}{Replication pipeline}
\small
\begin{itemize}
  \item Data: OpenICPSR package in
  \texttt{data/raw/openicpsr\_wu2018\_replication-pkg/}.
  \item Command: \texttt{python src/run\_all.py}.
  \item Outputs:
  \texttt{output/figures/figure1.pdf},
  \texttt{output/tables/table1.csv},
  \texttt{output/tables/table2.csv}, and Lasso intermediates.
\end{itemize}
\end{frame}

\begin{frame}{Why do the numbers match exactly?}
\begin{itemize}
  \item We used the exact same OpenICPSR dataset as Wu (2018).
  \item We did not modify any preprocessing steps.
  \item The same train/test splits were used as in the original code.
  \item The model specification (Lasso logit) and hyperparameters were unchanged.
  \item The original random seed was preserved.
\end{itemize}
\medskip
\small
Because the original replication code was run on the same data with identical settings, we expect the coefficient estimates and marginal effects to match exactly (up to rounding).
\end{frame}

\begin{frame}{Obstacles and fixes}
\begin{itemize}
  \item Pandas deprecation: replaced \texttt{as\_matrix()} with \texttt{to\_numpy()}.
  \item NumPy security change: added \texttt{allow\_pickle=True} to \texttt{np.load(...)}.
  \item These changes restore compatibility only; analysis logic unchanged.
\end{itemize}
\end{frame}

\begin{frame}{Results: Table 1 (full sample)}
\centering
\scriptsize
\begin{adjustbox}{max width=0.95\linewidth}
\begin{tabular}{l r l r}
\toprule
\multicolumn{2}{c}{Most female} & \multicolumn{2}{c}{Most male} \\
\cmidrule(lr){1-2} \cmidrule(lr){3-4}
Word & ME & Word & ME \\
\midrule
Hotter & 0.422 & Homo & -0.303 \\
Pregnant & 0.323 & Testosterone & -0.195 \\
Plow & 0.277 & Chapters & -0.189 \\
Marry & 0.275 & Satisfaction & -0.187 \\
Hot & 0.271 & Fieckers & -0.181 \\
Marrying & 0.260 & Macroeconomics & -0.180 \\
Pregnancy & 0.254 & Cuny & -0.180 \\
Attractive & 0.245 & Thrust & -0.169 \\
Beautiful & 0.240 & Nk & -0.165 \\
Breast & 0.227 & Macro & -0.163 \\
\bottomrule
\end{tabular}

\end{adjustbox}
\medskip
\footnotesize
\vspace{0.1em}
\textit{\\
Note: Results exactly match Wu (2018) up to rounding, as expected when running the original replication code on the same dataset.}
\end{frame}

\begin{frame}{Results: Table 2 (pronoun sample)}
\centering
\scriptsize
\begin{adjustbox}{max width=0.95\linewidth}
\begin{tabular}{l r l r}
\toprule
Female word & ME (pronoun) & Male word & ME (pronoun)\\
\midrule
pregnancy & 0.292 & knocking & -0.329\\
hotter & 0.289 & testosterone & -0.204\\
pregnant & 0.258 & blog & -0.183\\
hp & 0.238 & hateukbro & -0.176\\
vagina & 0.228 & adviser & -0.175\\
breast & 0.220 & hero & -0.174\\
plow & 0.219 & cuny & -0.173\\
shopping & 0.207 & handsome & -0.166\\
marry & 0.207 & mod & -0.166\\
gorgeous & 0.201 & homo & -0.160\\
\bottomrule
\end{tabular}

\end{adjustbox}
\medskip
\vspace{0.1em}
\textit{\\
Note: Also matches!}
\end{frame}

\begin{frame}{Results: Figure 1}
\centering
% Use height (not width) so the full figure fits comfortably on the slide.
\includegraphics[height=0.72\textheight,keepaspectratio]{../../output/figures/figure1.pdf}

\vspace{0.15em}
\tiny
\setlength{\tabcolsep}{4pt}
\renewcommand{\arraystretch}{0.9}
\begin{tabular}{@{}ll\hspace{0.8em}ll@{}}
  \begin{tikzpicture}[baseline=-0.4ex]
    \draw[maleBlue, thick] (0,0) -- (0.55,0);
  \end{tikzpicture} & Male (all classified posts)
  & \begin{tikzpicture}[baseline=-0.4ex]
    \draw[femalePink, thick] (0,0) -- (0.55,0);
  \end{tikzpicture} & Female (all classified posts) \\
  \begin{tikzpicture}[baseline=-0.4ex]
    \draw[maleBlue, thick, dashed] (0,0) -- (0.55,0);
  \end{tikzpicture} & Male (pronouns only)
  & \begin{tikzpicture}[baseline=-0.4ex]
    \draw[femalePink, thick, dashed] (0,0) -- (0.55,0);
  \end{tikzpicture} & Female (pronouns only) \\
\end{tabular}
\end{frame}

\begin{frame}{Conclusion and next steps}
\begin{itemize}
  \item Replication targets were reproduced with a single-command pipeline.
  \item Next step (extension): re-estimate Table 1 using an alternative prediction model (e.g., OLS on word counts) and compare to Lasso.
\end{itemize}
\end{frame}

\end{document}
